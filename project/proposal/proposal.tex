\documentclass[11pt]{article}

\usepackage[margin=1in]{geometry}
\usepackage{setspace}
\usepackage{hyperref}

\title{\textbf{Predicting Racketlon Match Score Differences Using Discipline-Specific Elo Ratings and Machine Learning}}
\author{Zain Magdon-Ismail}
\date{}

\begin{document}
\maketitle

\onehalfspacing

\section*{Abstract}
Racketlon is a multi-discipline sport in which match outcomes are determined by cumulative point differences across four racket sports: table tennis, badminton, squash, and tennis. This project aims to develop a data-driven system to predict discipline-level score differences using historical match data and engineered performance features. A time-aware, discipline-specific Elo rating system will be constructed to model player skill over time. These ratings, along with contextual and recent-form features, will be used in supervised learning models to estimate per-discipline score margins, which are then aggregated to predict overall match outcomes.

\section*{Problem Statement}
Predicting outcomes in multi-discipline sports presents unique challenges due to heterogeneous player strengths across disciplines and temporal variation in performance. Binary win-loss prediction fails to capture the margin-based structure that determines match results in racketlon, where cumulative points decide the winner. This project addresses this limitation by modeling score differences at the discipline level, allowing for a more granular and informative representation of match dynamics. The project aligns with course concepts in supervised learning, regression, feature engineering, and model evaluation.

\section*{Proposed Approach and Techniques}
Match prediction will be formulated as a supervised regression problem at the discipline level. For each match, separate regression targets will be defined corresponding to score differences in table tennis, badminton, squash, and tennis. Predicted discipline-level margins will be aggregated to estimate overall match score difference and inferred match outcomes.

A discipline-specific Elo-style rating system will be implemented to estimate player strength prior to each match, incorporating temporal ordering and margin-aware updates. Elo-derived features will serve as both a strong baseline and core model inputs. Additional features capturing recent form, head-to-head history, experience, and match context will be engineered using only information available prior to each match.

Regression models such as linear regression and tree-based methods will be trained to predict discipline-level score differences. Model performance will be evaluated using error-based metrics at both the discipline and match levels, and analysis will be conducted to understand which factors most strongly influence prediction accuracy.

\section*{Data and Resources}
The dataset consists of historical racketlon match data scraped from \href{https://fir.tournamentsoftware.com}{fir.tournamentsoftware.com}. Available data includes match dates, tournament context, player identifiers, and per-discipline scores. All features will be computed using strictly pre-match information to prevent temporal leakage. Data processing and modeling will be conducted in Python using standard data science libraries. GPU resources are not expected to be required.

\section*{Expected Outcomes}
The expected outcomes of this project include:
\begin{itemize}
    \item Meaningful discipline-specific and overall player ratings
    \item Accurate prediction of score differences at both the discipline and match levels
    \item Analytical insights into the effects of discipline imbalance, recent form, and experience on match outcomes
\end{itemize}

\section*{Team Roles and Collaboration Plan}
This project will be completed individually. All aspects of data preprocessing, feature engineering, modeling, evaluation, and documentation will be handled independently.
\section*{Timeline}
\begin{itemize}
    \item \textbf{January--February}: Data preprocessing, exploratory analysis, and rating system implementation (Done Mostly)
    \item \textbf{February--March}: Feature engineering and regression model development
    \item \textbf{March--April}: Model evaluation, error analysis, and refinement
    \item \textbf{April--May}: Final report writing and optional extensions
\end{itemize}

\end{document}
